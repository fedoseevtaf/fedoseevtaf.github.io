\documentclass[oneside, final, 14pt]{extreport}
\usepackage[utf8]{inputenc}
\usepackage[russianb]{babel}
\usepackage{vmargin}
\setpapersize{A4}
\setmarginsrb{2cm}{1.5cm}{2cm}{1.5cm}{0pt}{0mm}{0pt}{13mm}
\usepackage{indentfirst}
\usepackage{amsmath}
\usepackage{amssymb}
\usepackage{xcolor}

\definecolor{customblue}{RGB}{40, 116, 166}

\usepackage[
	colorlinks=true,
	urlcolor=customblue,
	linkcolor=green,
	citecolor=blue,
	anchorcolor=yellow
]{hyperref}
\usepackage{graphicx}

\newcommand\Par[1]{

#1

}

\newcommand\Blockquote[1]{
#1
}

\newcommand\Title[1]{\Par{
{\Huge
\noindent
\textbf{#1}
}}}

\newcommand\TitleSub[1]{\Par{
{\Large
\noindent
\textit{#1}
}}}


\newcommand\Section[1]{\Par{
{\large
\begin{centering}
\noindent
\textbf{#1}

\end{centering}
}}}

\newcommand\Subsection[1]{\Par{
{\large
\noindent
#1
}}}

\newcommand\SubsectionSub[1]{\Par{
\noindent
\textit{#1}
}}

\newcommand\InlineImg[1]{
\(
\begin{array}{l}
\includegraphics[height=2em]{#1}
\end{array}
\)
}



\begin{document}
\begin{titlepage}

\BgThispage

\Title{Федосеев Трофим Андреевич}
\TitleSub{студент}

\begin{flushright}
\href[pdfnewwindow=true]{https://t.me/av10nyt}{
\InlineImg{misc/telegram-logo}
Telegram: @av10nyt
}

\href[pdfnewwindow=true]{https://github.com/fedoseevtaf/}{
\InlineImg{misc/github-logo}
GitHub: fedoseevtaf
}

\end{flushright}

\Section{Навыки}

\begin{centering}
\texttt{Python, C, x86 assembly}

\end{centering}

\Section{Образование}

\Blockquote{
\Subsection{МГУ (в процессе)}
\SubsectionSub{2024 - настояшее время, бакалавриат}
}

Я учусь на певом кусре факультета Вычислительной Математики и
Кибернетики МГУ.

Во время обучения разработал учебный проект, в котором
компоненты написанные на Си взаимодействуют с компонентами,
написанными на языке ассемблера.

\Section{Мои проекты}

\Blockquote{
\Subsection{mj}
}
Mj - это микро-тестирующая система, утилита для автоматического
запуска тестов. Она разработана для таких же студентов, как я;
утилита помогает при решении алгоритмических задач во время обучения.
Гитхаб репозиторий:
\href{https://github.com/fedoseevtaf/mj}{fedoseevtaf/mj}

\Blockquote{
\Subsection{nastx}
}
Nastx - это библиотека функций, печатающих состояние x87
сопроцессора. Я разрабатываю её чтобы упростить отладку,
для студентов, которые изучают набор интрукций x87.
Гитхаб репозиторий:
\href{https://github.com/fedoseevtaf/nastx}{fedoseevtaf/nastx}

\clearpage
\end{titlepage}
\end{document}


